\documentclass{article} % For LaTeX2e
\usepackage{nips13submit_e,times}
\usepackage{hyperref}
\usepackage{url}
%\documentstyle[nips13submit_09,times,art10]{article} % For LaTeX 2.09


\title{Project Proposal: Digit Recognizer}

\author{
Jimmy Lin\thanks{ Use footnote for providing further information
about author (webpage, alternative address)---\emph{not} for acknowledging
funding agencies.} \\
Department of Computer Science\\
University of Texas at Austin\\
Austin, TX 78712 \\
\texttt{JimmyLin@utexas.edu} \\
%{{{ For More Author
\And
Coauthor \\
Affiliation \\
Address \\
\texttt{email} \\
%\AND
%Coauthor \\
%Affiliation \\
%Address \\
%\texttt{email} \\
%\And
%Coauthor \\
%Affiliation \\
%Address \\
%\texttt{email} \\
%\And
%Coauthor \\
%Affiliation \\
%Address \\
%\texttt{email} \\
%(if needed)\\
%}}}
}

% The \author macro works with any number of authors. There are two commands
% used to separate the names and addresses of multiple authors: \And and \AND.
%
% Using \And between authors leaves it to \LaTeX{} to determine where to break
% the lines. Using \AND forces a linebreak at that point. So, if \LaTeX{}
% puts 3 of 4 authors names on the first line, and the last on the second
% line, try using \AND instead of \And before the third author name.

\newcommand{\fix}{\marginpar{FIX}}
\newcommand{\new}{\marginpar{NEW}}

\nipsfinalcopy % Uncomment for camera-ready version


% TODO: Find a report guideline 
% TODO: 
\begin{document}


\maketitle

\begin{abstract}
    BACKGROUND. The idea of recognizing digit . 
    The goal in this project is to take an image of a handwritten single
    digit, and determine what that digit is.  
    
    Kaggle competition. 
    We propose to 

\end{abstract}

\section{Introduction}

\section{Motivations}


\section{Project Details}
\subsection{Provided Baseline Models}
As mentioned above, implementations of two baseline models are provided by the
holder of that kaggle competition. These baselines are respectively K-Nearest
Neighbour Model and Random Forest Model. The following subsection will present
reader the basic introduction of two baseline model.  

\subsection{Models to Explore}
Our current plan is to further investigate the performance of a series of machine learning
algorithm on.

\section{Conclusion}

\section{References}

\small{
[1] Alexander, J.A. \& Mozer, M.C. (1995) Template-based algorithms
for connectionist rule extraction. In G. Tesauro, D. S. Touretzky
and T.K. Leen (eds.), {\it Advances in Neural Information Processing
Systems 7}, pp. 609-616. Cambridge, MA: MIT Press.

[2] Bower, J.M. \& Beeman, D. (1995) {\it The Book of GENESIS: Exploring
Realistic Neural Models with the GEneral NEural SImulation System.}
New York: TELOS/Springer-Verlag.

[3] Hasselmo, M.E., Schnell, E. \& Barkai, E. (1995) Dynamics of learning
and recall at excitatory recurrent synapses and cholinergic modulation
in rat hippocampal region CA3. {\it Journal of Neuroscience}
{\bf 15}(7):5249-5262.
}

\newpage
\section{Appendix}
\subsection{PCA}
\subsection{Neural Network}
\subsection{Multi-class Support Vector Machine}
\subsection{Deep Learning}

\end{document}
