\documentclass{article} % For LaTeX2e
\usepackage{nips13submit_e,times}
\usepackage{hyperref}
\usepackage{url}
%\documentstyle[nips13submit_09,times,art10]{article} % For LaTeX 2.09


\title{Project Proposal: Digit Recognizer}

\author{
Jimmy Lin (xl5224) \\
Department of Computer Science\\
University of Texas at Austin\\
Austin, TX 78712 \\
\texttt{jimmylin@utexas.edu} \\
\And
Jisong Yang (jy7226) \\
Department of Statistical Scientfic Computation\\
University of Texas at Austin\\
Austin, TX 78712 \\
\texttt{jsyang993@gmail.com} \\
%{{{ For More Author
%\AND
%Coauthor \\
%Affiliation \\
%Address \\
%\texttt{email} \\
%\And
%Coauthor \\
%Affiliation \\
%Address \\
%\texttt{email} \\
%\And
%Coauthor \\
%Affiliation \\
%Address \\
%\texttt{email} \\
%(if needed)\\
%}}}
}

% The \author macro works with any number of authors. There are two commands
% used to separate the names and addresses of multiple authors: \And and \AND.
%
% Using \And between authors leaves it to \LaTeX{} to determine where to break
% the lines. Using \AND forces a linebreak at that point. So, if \LaTeX{}
% puts 3 of 4 authors names on the first line, and the last on the second
% line, try using \AND instead of \And before the third author name.

\newcommand{\fix}{\marginpar{FIX}}
\newcommand{\new}{\marginpar{NEW}}

\nipsfinalcopy % Uncomment for camera-ready version

\begin{document}

\maketitle

%\begin{abstract}
%    BACKGROUND. The idea of recognizing digit . 
%    The goal in this project is to take an image of a handwritten single
%    digit, and determine what that digit is.  
%    
%    Kaggle competition. 
%    We propose to 

%\end{abstract}

\section{Introduction}
% MOTIVATION SENTENCE
Recognizing hand-written characters has long been a hot topic in artificial
intelligence community. 
% KEY CONCEPT (PROBLEM) ILLUMINATION
Handwriting recognition is the ability of a computer to receive and
interpret intelligible handwritten input from sources such as paper documents,
photographs, touch-screens and other devices.
% SUMMARIZE THE SOLUTION
In the past decades, many probabilistic or non-probabilistic algorithms have been exploited to
effectively solve this problem, including K-nearest neighbour,
random forest, and well-known neural network. 

% DESCRIBE FORMAT OF REST OF PROPOSAL
In this project proposal, we will explain the reason of diving into this
problem and demonstrate reader some details of our experimental project.
And more importantly, the planned timeline of project progress and our
expectation will be present to reader in later section.

\section{Motivation}

% APPLICATION OF THIS PROBLEM 
As to the historical application of handwriting recognition, 
it can be traced back to early 1980s.  There are plenty of commercial products
incorporating handwriting recognition as a replacement for keyboard input were
introduced. The hardware application continued to develop
in the following decades.  Until the recent years, Tablet PCs can be regarded
as a special notebook computer that is outfitted with a digitizer tablet and a
stylus, and allows a user to handwrite text on the unit's screen. In addition,
highly efficient handwriting recognizers were developed in real world to apply
on the zip codes detection.

% RELATED WORKS/SOLUTION
Plenty of mature solutions has already been developed to existence in solving
the handwriting problem. 
Since 2009, the recurrent neural networks and deep feedforward neural networks
developed in the research group of Jürgen Schmidhuber at the Swiss AI Lab
IDSIA have outshined many other models in competitions.
% POSSIBLE IMPROVEMENTS?
The fact that existing algorithms are powerful in recognizing digits does not
remove possibility of further improvement. The chance of improvement lies in
reduction of computational complexity for recognizing digits in higher degree.


% \section{Project Summary}

\section{Project Details}

What deserves being mentioned is that the holder of this Kaggle Competition
provides us two basic models to better illuminate the problem of digit
recognition and the form of given dataset, the MINIST.

\subsection{Provided Baseline Models}
As mentioned above, implementations of two baseline models are provided by the
holder of that kaggle competition. These baselines are respectively K-Nearest
Neighbour Model and Random Forest Model. However, the baseline models are
extremely computationally expensive since they naively employ the raw data
without turning to feature extraction methods. 

\subsection{Models To Explore}
In this project, we should first implement effective feature extraction from
the raw image intensity data in MINIST. One possible method for attempt is
Principle Component Analysis, which extracts the most distinct components of
specified size. 

Besides, we will further investigate the performance of a series of machine learning
algorithms on digit recognition. Our attempt for this part will involve in
neural network algorithm, Support Vector Machine with multiple class labeling,
and treat Deep Learning algorithm as extension. For specific implementation of
other algorithms, we will refer to existing libraries.

\subsection{Timeline}
% How this project is arranged
The project timeline is shown as follows:
\begin{itemize}
    \item{\textbf{March. 23 - April. 02}. Propose project. Run two 
            provided baseline models and evaluate its performance. 
            Implement feature extraction PCA.}
    \item{\textbf{April. 02 - April. 12}. Explore performance of new models in
        given dataset. Tasks may involves in reading relevant paper. Current
        plan at this stage is to achieve neural network algorithm and
        support vector machine on the preprocessed data.}
    \item{\textbf{April. 13 - April. 18}. Draft initial version of project report
        . Record and compare temporal and spatial perfermance of all attempted algorithms. }
    \item{\textbf{April. 19 - April. 25}. Extension: Further explore some more advanced
            models, say, Deep neural network. (optional)}
    \item{\textbf{April. 25 - May. 2}. Put down new progress at second-time
            exploration and proofread the project report.}
\end{itemize}

% task assignment
The detailed task assignments are shown as follows:

    Jisong Yang: modify project proposal, PCA, Neural Network, DNN(optional),
    final project report \\
    Jimmy Lin: draft project proposal, SVM, DNN(optional), final project report

\section{Conclusion} \label{Conclusion}
    % primary objective 
    The main objective of this project is to apply various existing machine
    learning algorithms on the long-lasting digit recognition problem and
    evaluate the performance of each algorithm on solving this particular
    problem.
    % secondary objective
    In addition, we leave another problem, the feature extraction, as our
    secondary focus. 
    
    % expectation of result
    Our wish is to achieve a model with digit recognition accuracy as close to
    $100\%$ as possible. Actually, about $10$ teams have already fulfill
    perfect recognizer, identifying given digit image with no error!  But for
    us, a accuracy over $90\%$ is acceptable for this experimental project.

\section{References}

\small{
% FEATURE EXTRACTION
[1] Liu, C. L., Nakashima, K., Sako, H., \& Fujisawa, H. (2003). Handwritten
digit recognition: benchmarking of state-of-the-art techniques. {\it Pattern
    Recognition}, {\bf 36(10)}, 2271-2285.

% NEURAL NETWORK
[2] Le Cun, B. B., Denker, J. S., Henderson, D., Howard, R. E., Hubbard, W., \&
Jackel, L. D. (1990). Handwritten digit recognition with a back-propagation
network. In {\it Advances in neural information processing systems.}

% NEURAL NETWORK
[3] LeCun, Y., Jackel, L. D., Bottou, L., Brunot, A., Cortes, C., Denker, J.
S., \& Vapnik, V. (1995, October). Comparison of learning algorithms for
handwritten digit recognition. In {\it International conference on artificial
    neural networks} {\bf (Vol. 60)}.

% THEORY: DEEP LEARNING PAPER
[4] Hinton, G. E., Osindero, S., \& Teh, Y. W. (2006). A fast learning
algorithm for deep belief nets. {\it Neural computation}, {\bf 18(7)}, 1527-1554.

% APPLICATION: DEEP LEARNING ON DIGIT RECOGNITION
[5] Ciresan, D. C., Meier, U., Gambardella, L. M., \& Schmidhuber, J. (2010).
Deep, big, simple neural nets for handwritten digit recognition. {\it Neural
    computation}, {\it 22(12)}, 3207-3220.
}

\end{document}
